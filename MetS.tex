\documentclass[11pt]{beamer}
\usetheme{m}

\usepackage{amsmath}
\usepackage{amsfonts}
\usepackage{amssymb}
\usepackage{graphicx}
\usepackage{booktabs}
\usepackage{hyperref}
\usepackage{multicol}
\graphicspath{{F:/Official/Talks/BariatricJIPMER_2016/Assets/Images/}}
\usepackage[style=numeric,backend=biber,doi=false,url=false,isbn=false]{biblatex}
\addbibresource{MetS.bib}
\author{Karthik Balachandran}
\title{Metabolic syndrome and bariatric surgery}
\subtitle{An Endocrine perspective}
%\logo{}
%\institute{}
\date{$20^{th}$ October 2016}
%\subject{}
%\setbeamercovered{transparent}
\setbeamertemplate{navigation symbols}{}

\begin{document}
	\maketitle
	
	\begin{frame}
		\frametitle{Agenda}
\begin{itemize}
  \item Problem - an endocrine view
  \item Role of endocrinologist
  \item Shared challenges and opportunities
\end{itemize}
	\end{frame}

\section{Problem}
\begin{frame}{Defining Metabolic syndrome }
\includegraphics[width=\linewidth]{Definitions.png}
\end{frame}	

{
\setbeamertemplate{navigation symbols}{}
\setbeamertemplate{background canvas}{\includegraphics[height = \paperheight, 
                                      width = \paperwidth]{maleobesity.png}}
\begin{frame}[plain]
\end{frame}
} 

{
\setbeamertemplate{navigation symbols}{}
\setbeamertemplate{background canvas}{\includegraphics[height = \paperheight, 
                                      width = \paperwidth]{femaleobesity.png}}
\begin{frame}[plain]
\end{frame}
} 

%----------------------------------------------------------------------------------------------------------------------------
%---------------------------------------------------------------------------------------------------------------------------- 
{
\setbeamertemplate{navigation symbols}{}
\setbeamertemplate{background canvas}{\includegraphics[height = \paperheight, 
                                      width = \paperwidth]{"YY_paradox.jpg"}}
\begin{frame}[plain]
\end{frame}
}
%----------------------------------------------------------------------------------------------------------------------------
%---------------------------------------------------------------------------------------------------------------------------- 
\begin{frame}{The scale}
In the USA alone, it is estimated that it would take 5500 surgeons doing 400 cases per year, each for 10 years to treat the 22 million obese Americans\footfullcite{Kurian.2016}.
\end{frame} 
%%%%%%%%%%%%%%%%%%%%%%%%%%%%%%%%%%%%%%%%%%%%%%%%%%%%%%%%%%%%%%%%%%%%
\begin{frame}{Awareness / Attitude }
\includegraphics[scale=0.5]{vignette.png}
\end{frame} 
%---------------------------------------------------------------------------------------------------------------------------- 
\begin{frame}{ Awareness /Attitude }
\includegraphics[scale=0.5]{options.png}
\end{frame} 
%---------------------------------------------------------------------------------
\section{Guidelines}
\begin{frame}{Evolution of thought  }
\includegraphics[width=\linewidth,height=0.5\textheight]{evolution.png}
\end{frame} 
%---------------------------------------------------------------------------------- 
\begin{frame}{Bariatric surgery vs Metabolic surgery  }
\begin{itemize}

\item Baros = weight
\item Metabolic surgery - done with the intent of treating some metabolic disorder
\item Increasing shift -American society of bariatric surgery has changed its name to reflect this
	
\end{itemize}
\end{frame} 
%---------------------------------------------------------------------------------- 
\begin{frame}{DSS 2  }
$ \oplus $ Surgery should be \emph{considered} for patients with type2 DM and a BMI of \emph{$  30 - 34.9 kg/m^{2}$}, if hyperglycemia is inadequately controlled despite optimal medical therapy \\ [1 ex]
$ \oplus $ These thresholds should be reduced by $ 2.5 kg/m^{2} $ for Asians \footfullcite{Rubino.2016}
\end{frame} 

%---------------------------------------------------------------------------------- 
\begin{frame}{Rationale for the use of surgery  }
\begin{itemize}

\item Non weight loss mediated mechanism \pause
    \begin{itemize}
        \item Changes in gut hormones (incretins and decretins,FGF19) 
        \item Changes in bile acid metabolism
        \item Improvement in microbiota
        \item Improvement in nutrient sensing
    \end{itemize}
\item Cost effectiveness - one time therapy \pause
\item Efficacy - no medical therapy is known to induce remission \pause
\item Long term data on safety 	
\end{itemize}
\end{frame} 
%---------------------------------------------------------------------------------- 
\begin{frame}{Evidence - surgery vs medicine  }
 \begin{figure}
 \includegraphics[scale=0.5]{boxplot}
 \caption{Overall change in HbA1c in 11 RCTs\footfullcite{Rubino.2016}}
 \end{figure}
\end{frame} 
%---------------------------------------------------------------------------------- 
\begin{frame}{Evidence across BMI categories  }
 \begin{figure}
 \includegraphics[scale=0.6]{lowvshigh.jpg}
 \caption{Bariatric surgery across BMI categories\footfullcite{Cummings.2016}}
 \end{figure}
\end{frame} 
%---------------------------------------------------------------------------------- 
\begin{frame}{Where does metabolic surgery stand in the management of diabetes?  }
\begin{itemize}

\item No current treatment algorithm includes a role for surgical intervention
\item Role is being increasingly recognized
\item Shifting eligibility criteria and international ratification by medical bodies
	
\end{itemize}

\end{frame} 
%---------------------------------------------------------------------------------- 
{
\setbeamertemplate{navigation symbols}{}
\setbeamertemplate{background canvas}{\includegraphics[height = \paperheight, 
                                      width = \paperwidth]{algorithm.png}}
\begin{frame}[plain]
\end{frame}
} 
%---------------------------------------------------------------------------------- 
\section{Medical / Endocrine management}
\begin{frame}{Obesity - effects }
  \begin{figure}
  \includegraphics[scale=0.7]{endochanges}
  \end{figure}
\end{frame} 
%----------------------------------------------------------------------------------
\begin{frame}{ Endocrine issues }
\begin{itemize}

\item Subclinical hypothyroidism
\item Polycystic ovarian disease
\item Pseudocushing's syndrome
\item Vitamin D deficiency
\item Secondary hyperparathyroidism
\item Insulin resistance $\rightarrow  $ Acromegaloid appearance
\item Hyperuricemia
\end{itemize}
\end{frame}  
%---------------------------------------------------------------------------------- 
\begin{frame}[shrink]{Lab monitoring }
   \begin{flushleft}
   \begin{figure}
   \includegraphics{labs}
   \end{figure}
   \end{flushleft}
\end{frame}
%---------------------------------------------------------------------------------- 
 \begin{frame}{Calcium  }
 \begin{itemize}
 
 \item Routine supplementation after 6 months
 \item 1200 mg/d
 \item Calcium citrate preferred as it doesn't required acid
 	
 \end{itemize}
 \end{frame} 
 %---------------------------------------------------------------------------------- 
 \begin{frame}{Vitamin B12  }
  \begin{itemize}
  
  \item Initiation within 6 months
  \item Optimal dosing not known
  \item Oral dosing of 350 mcg/day maintains normal levels
  	
  \end{itemize}
 \end{frame} 
 %---------------------------------------------------------------------------------- 
 \begin{frame}{Folate  }
 \begin{itemize}
 
 \item Routine supplementation after 6 months
 \item 400 mcg/day
 	
 \end{itemize}
 \end{frame}
 %---------------------------------------------------------------------------------- 
 \begin{frame}{ Iron }
 \begin{itemize}
 
 \item Interaction with calcium
 \item Poor meat intake contributes
 \item Deficiency can be prolonged
 	
 \end{itemize}
 \end{frame} 
 %---------------------------------------------------------------------------------- 
 \begin{frame}{ Vitamins }
 \begin{itemize}
 
 \item Vitamin D as RDA- 60000 IU every two months orally.
 \item Vitamin K not routinely needed. Supplement if INR > 1.4
 \item Vitamin B1 supplementation in patients with intractable vomiting
 	
 \end{itemize}
 \end{frame} 
 %---------------------------------------------------------------------------------- 
 \begin{frame}{Diabetes  }
 \begin{itemize}
 
 \item HbA1c < 7 \% goal
 \item FBS <110 mg/dl
 \item PPBS <180 mg/dl
 	
 \end{itemize}
 \end{frame} 
 %---------------------------------------------------------------------------------- 
 \begin{frame}{Pregnancy and fertility  }
 \begin{itemize}
 
 \item Better avoided for 12 to 18 months \footfullcite{Martin.2013}
 \item Important in PCOD patients
 	
 \end{itemize}
 \end{frame} 
 %---------------------------------------------------------------------------------- 
 \begin{frame}{Lipids  }
 \begin{itemize}
 
 \item Managed as per NCEP ATP III guidelines
 \item Frequent monitoring required
 	
 \end{itemize}
 \end{frame}
 %---------------------------------------------------------------------------------- 
 \begin{frame}{ Gout }
 \begin{itemize}
 
 \item Higher risk if BMI >40
 \item Rapid fall in uric acid levels contribute
 \item Prophylactic therapy may be needed depending on uric acid levels
 	
 \end{itemize}
 \end{frame}
 %---------------------------------------------------------------------------------- 
 \begin{frame}{Dumping syndrome  }
 \begin{itemize}
 
 \item Rapid passage of stomach contents into small intestine	
 \item Non pharmacological measures 
 \begin{itemize}
 \item Small frequent meals
\item Avoiding ingestion of liquids within half an hour of solid meal
\item Avoiding simple sugars
\item Increasing protein intake

 \end{itemize}
 \item If unsuccessful, octreotide 30 min before food
 	
 \end{itemize}
 \end{frame} 
 %---------------------------------------------------------------------------------- 
 \begin{frame}{Postprandial hypoglycemia  }
 \begin{itemize}
 \item May present 2 - 9 years after RYGB
 \item Nesidioblastosis vs inappropriate insulin kinetics
 \end{itemize}
 \end{frame} 
 
 %---------------------------------------------------------------------------------- 
 \begin{frame}[shrink]{Medical management in a nutshell  }
 \includegraphics{nutrition}
 \end{frame} 
 %---------------------------------------------------------------------------------- 
 \begin{frame}{ What do we do? }
 \begin{itemize}
 
 \item Frank discussion with the patient on the limits of therapies on their costs
 \item Stress that bariatric surgery is not a replacement for culinary discipline
 \item Provide appropriate medical care(and avoid unnecessary medication) for any hormonal disorders
 	
 \end{itemize}
 \end{frame} 
 
 %---------------------------------------------------------------------------------- 
  \section{Challenges and opportunities}
  \begin{frame}{What can we do together  }
   \begin{itemize}
   
   \item Going bold like BOLD \pause
   \item Frugal innovation to bring down the cost \pause
   \item Education and awareness \pause
   \item Training and accreditation \pause
   \item Rethinking cut offs \pause
   \item Rebranding as metabolic surgery \pause
   \item Mentorship and training programmes - regional/national leadership
   	
   \end{itemize}
  \end{frame} 
 %---------------------------------------------------------------------------------- 
 \begin{frame}{  }
     \centering \Huge {Thank You}
 \end{frame} 
\end{document} 